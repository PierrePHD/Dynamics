
\documentclass[12pt,a4paper]{report}
  \usepackage[utf8]{inputenc}
  \usepackage[francais]{babel}
  \usepackage[T1]{fontenc}
%\usepackage{mdwlist}								% for a compact itemize*
  
  \begin{document}
  
La méthode LATIN (Large Time INcrement method) [Ladevèze, 1985, Ladevèze, 1989, Ladevèze, 1999] a été initialement proposée pour traiter des problèmes non li- néaires dépendants du temps. Depuis son introduction, cette méthode a été exploitée autour de nombreux grands axes de développements :
  \begin{itemize}
  \item Grandes déformations,plasticité(PGD)[Bussy,1984,Vauchez,1991,Liu,1992]
  \item Grandes transformations,formulation corotationnelle (PGD) [Boucard et al.,1997, Aubard et al., 2002]
  \item Elasto-visco-plasticité, grand nombre de cycles, thermo-élasto-visco-plasticité (PGD) [Cognard, 1989, Arzt, 1994, Cognard et al., 1999]
  \item Dynamiquenon-llinéaire(PGD)[Royer,1990]
  \item Endommagement et composites
  	\begin{itemize}
  		\item Formulation fonctionnelle (loi de comportement intégrale) [Allix and Lade- vèze, 1992, Guinard et al., 2002]
	 	\item Formulationvariablesinternes[DouchinandLadevèze,2001]  \item Multi-échellecomposite[Trovalet,2010]
	 	\item Couplageflambage-délaminage[Saavedraetal.,2012]
  	\end{itemize}  
  	\item Dynamiquerapide[Lemoussuetal.,2002,Boucardetal.,2003,Guptaetal.,2005]
    \item Assemblages
    
		\begin{itemize}
		 \item 2Dstatique[Danwe,1993]
		 \item Evolutionversle3D(bridesenstatique)[Champaneyetal.,1999]  \item Applicationenquasi-statique[Blanzéetal.,2000]
		 \item Extensionàladynamique[Lemoussuetal.,2002]
		 \item Optimisation[Boucardetal.,2010,Laurentetal.,2013]
 		\end{itemize}
 	\item Décompositiondedomaineausenslarge
		\begin{itemize}
		 \item Parallélisme[Ladevèze,1987,Lorong,1994,Champaneyetal.,1997]
		 \item Multiéchelle[LadevèzeandDureisseix,1999,Ladevèzeetal.,2001]
		 \item Multiéchelle-stratégieréutilisation,stochastique[Ladevèzeetal.,2002]  \item Multiphysique[Dureisseixetal.,2003]
		 \item CouplageXFem-fissuration[Guidaultetal.,2007]
		 \item Multiéchelle3D[Ladevèzeetal.,2006]
		 \item Multiéchelletemps[Ladevèzeetal.,2010]
		 \item Amortissement[Caignotetal.,2010]
		 \item Composites[Allixetal.,2010]
 		\end{itemize}
 	\item Identification-Problèmesinverses
		\begin{itemize}
		 \item IdentificationavecPGD[AllixandVidal,2002]
		 \item CouplageavecRICATI[Nguyenetal.,2008]
 		\end{itemize}
 	\item MultiparamétriquesansPGD
		\begin{itemize}
		 \item Assemblages[BoucardandChampaney,2003]
		 \item Dynamiquemultiéchelle[Boucardetal.,2011]
		 \item Assemblagecomposites[Rouletetal.,2013]
 		\end{itemize}
 	\item MultiparamétriqueavecPGD[Relunetal.,2013]
  \end{itemize}
C'est suivant ce dernier grand axe de recherche que s'illustre cette thèse.
De manière très grossière, on peut schématiser la méthode de résolution LATIN comme étant une stratégie qui génère une approximation de la solution s du problème sur l'ensemble de l'espace et du temps et en améliore automatiquement la qualité à chaque itération, quitte à commencer par une approximation très grossière. En ceci, la

\end{document}
