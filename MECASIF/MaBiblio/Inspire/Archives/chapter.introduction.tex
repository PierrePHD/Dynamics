\chapter{Introduction}
L'objectif de ce travail est d'appliquer une méthode de résolution PGD (Proper Generalized Decomposition), qui est une méthode de réduction de modèles où la solution est à variables séparées, à des problèmes de dynamique présentant des non-linéarités. Ce rapport propose ici introduction à la PGD avec une présentation d'autres méthodes de réductions de modèles, tel que la condensation, la méthode de Rayleigh-Ritz et la POD.

\section{La réduction de modèles}
La résolution numérique de problèmes de mécanique fait partie intégrante du travail de la plupart des ingénieurs, et connaît évidemment des limitations. Ces résolutions représentent des calculs gourmands en ressources et en temps, et la volonté croissante des industriels de s'appuyer le plus souvent sur la simulation plutôt que sur l'essai, place l'amélioration des méthodes utilisées au cœur de la problématique de nombreuses recherches. Pour l'industriel le gain de temps est un enjeu majeur pour être concurrentiel, voire pour réduire le coût d'une étude et donc au final de son produit. Les méthodes de réductions de modèles tendent à simplifier le problème pour réduire sa taille. Elles sont basées sur une diminution de la dimension de l'espace de recherche des solutions. Ceci nécessite la construction d'une
base réduite. On peut alors distinguer parmi ces méthodes, celles qui nécessitent une connaissance préalable d'une solution pour construire la base sur laquelle le problème sera projeté, elles sont dites "a posteriori", de celles dites "a priori" qui construisent la base pendant le calcul de résolution. 

Parmi les méthodes "a posteriori" se trouve la POD (Proper Orthogonal Decomposition), qui a fait l'objet de nombreux travaux d'études ces vingt dernières années. Il s'agit d'une méthode basée sur la séparation de variables qui sera présentée plus en détail par la suite. 

Outre une amélioration du temps de résolution pour des calculs couramment effectués par les industriels, la réduction de modèle peut se montrer si efficace dans la diminution du temps de calcul que de nouvelles résolutions peuvent être envisagées. En effet l'industrie rencontre des problèmes de tailles telles qu'elle ne peut envisager de les résoudre pour des questions de temps, voire de limites des performances même des plus puissantes machines. La première possibilité apportée par une réduction de modèle est l'utilisation pour l'optimisation. Puisque même si les méthodes "a posteriori" entraînent une première résolution du problème par une méthode standard, elles permettent, une fois la base réduite obtenue, de faire une série de calculs rapides pouvant permettre une optimisation ou une étude de type Monte Carlo sur le modèle réduit.

\section{La PGD}
\subsection{Généralités}
La PGD, pour "Proper Generalized Decomposition" , trouve son origine dans \cite{LATIN1} sous le nom de "radial time-space approximation", comme faisant partie la méthode LATIN (voir également \cite{LATIN2}). 

Parmi les possibilités apportées par la réduction de modèles, elle peut permettre de résoudre des problèmes jusque là inenvisageable par exemple dans la chimie quantique (quand des processus chimiques impliquent un nombre de molécules de réactifs si petit que le concept de continuité de concentration n'est plus valide, par exemple dans la recherche génétique). De tels problèmes peuvent rapidement être impossibles à résoudre par des méthodes standards, car ils souffrent de ce que l'on appelle la malédiction de la dimension. La solution est inaccessible par des approches traditionnelles de discrétisation utilisant un maillage car elles présentent une taille du système à résoudre qui croit exponentiellement en fonction du nombre de particules. Plus de détail sur ce sujet sont donné dans \cite{Paradigm}.

La diminution drastique de temps de calcul apportée par la PGD permet d'envisager de nouvelles utilisations de la simulation, notamment dans les DDDAS (pour Dynamic Data-Driven Application System), dont l'enjeu est de permettre à la simulation d'interagir avec une manipulation en temps réel. C'est-à-dire que les instruments actionneurs seront influencés par les résultats de calculs apportés par la simulation et les capteurs du système renverront des données à la simulation pour réajuster les calculs. Les DDDAS sont un sujet d'études identifié par la USNSF (United States National Science Foundation) dans son rapport en 2006 sur les SBES (Simulation-based Engineering Sciences) comme étant parmi les cinq challenges majeurs de la prochaine décennie. Pour plus de détails sur la PGD dans le cadre des DDDAS le lecteur pourra consulter \cite{DDAS}.

La PDG rencontre donc un franc succès depuis quelques années et ses applications sont pléthores. Un aperçu des différents travaux concernant cette méthode peut être trouvé dans \cite{ShortReview}.

\subsection{Mise en place de nouvelles variables : nouvelles coordonnées}
La PGD présente un autre avantage, par la mise en place de coordonnées supplémentaires, comme le décrivent \cite{Paradigm} et \cite{DDAS}. Il s'agit de rajouter au produit de fonctions à variables séparées, une fonction d'une nouvelle variable. Ceci permet de conserver une variable comme une inconnue et ainsi d'avoir un problème résolu quelque soit la valeur de cette variable. Pour obtenir un tel résultat sans cette méthode il faudrait effectuer de nombreux tirages de valeurs pour la variable et réaliser une approche de Monte Carlo, ce qui requerrait autant de résolutions. L'ajout de cette coordonnée permet donc de passer outre la nécessité de la valeur d'une variable au calcul, ou de résoudre une famille de problèmes qui dépendent de cette variable. C'est le principe de base de l'analyse stochastique, qui fait l'objet d'un intérêt grandissant. De cette manière la résolution du problème par la PGD apporte plus encore qu'une résolution directe, et peut permettre facilement la mise en place d'une optimisation. Dans les articles cités précédemment, les auteurs présentent comme exemple un problème de thermique à résoudre pour lequel la conductivité matériau est inconnue. Mais ceci ne se limite pas à une valeur matériau et peut s'appliquer à une condition initiale ou à une valeur géométrique...
