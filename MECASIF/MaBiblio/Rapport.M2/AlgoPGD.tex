\section{Résumé}

	
Séparation des variables:
\begin{equation}
	\displaystyle U_n(X,t) = \sum_{k=1}^n f_k(X) g_k(t)
\end{equation}

Formulation variationnelle:
\begin{equation}
\begin{array}{c}
 f_k = F (U_{k-1},g_k)\\
 g_k = G (U_{k-1},f_k)
\end{array}
\end{equation}

Algorithme
\begin{equation}{}
\begin{array}{l}
 for~k = 1$ à $ n \\
 \phantom{for~k} $initialiser $g_k \\
 \phantom{for~k} for~j=1$ à $j_{max} \\
 \phantom{for~k for~j } f_k = F (U_{k-1},g_k) \\
 \phantom{for~k for~j } g_k = G (U_{k-1},f_k) \\
 \phantom{for~k} end \\
end
\end{array}  \\
\end{equation}


\section{Orthogonalisation}

\subsection{Dans le point fixe}
\begin{equation}{}
\begin{array}{l}
 for~k = 1$ à $ n \\
 \phantom{for~k} $initialiser $g_k \\
 \phantom{for~k} for~j=1$ à $j_{max} \\
 \phantom{for~k for~j } f_k = F (U_{k-1},g_k) \\
 \displaystyle
 \phantom{for~k for~j } f_k = f_k - \sum_{i=1}^{k-1} (f_k.f_i)f_i\\
 \phantom{for~k for~j } f_k = f_k / norme(f_k)\\
 \phantom{for~k for~j } g_k = G (U_{k-1},f_k) \\
 \phantom{for~k} end \\
end
\end{array}  \\
\end{equation}

\subsection{Amélioration de la base}
\label{AlgorithmesPGDOrthoCorrect}
\begin{equation}{}
\begin{array}{l}
 for~k = 1$ à $ n \\
 \phantom{for~k} $initialiser $g_k \\
 \phantom{for~k} for~j=1$ à $j_{max} \\
 \phantom{for~k for~j } f_k = F (U_{k-1},g_k) \\
 \phantom{for~k for~j } g_k = G (U_{k-1},f_k) \\
 \phantom{for~k} end \\ 
 
 Commentaires
 \left\{\!
	 \begin{array}{l}
	 $Si on enlève une partie ainsi :$\\
	 \displaystyle
	 ~f_k = f_k - \sum_{i=1}^{k-1} (f_k.f_i)f_i\\
	 $On veux améliorer la base par :$\\
	 \displaystyle
	 ~U_{k-1} = U_{k-1} 
	 			+ \sum_{i=1}^{k-1} (f_k.f_i)f_i ~ g_k\\
	 $Une manière de le faire est d'impacter les $g_i$ :$\\
	 \displaystyle
	 ~U_{k-1} = \sum_{i=1}^{k-1} f_i g_i 
	 			+ \sum_{i=1}^{k-1} (f_k.f_i)f_i ~ g_k\\
	 \displaystyle
	 ~U_{k-1} = \sum_{i=1}^{k-1} f_i (g_i + (f_k.f_i)g_k)\\
	 $Donc :$\\
	 \end{array}
 \right.
 \\
 \phantom{for~k} for~i=1$ à $(k-1) \\
 \phantom{for~k for~i } g_i = g_i + (f_k.f_i)g_k\\
 \phantom{for~k} end\\
 \displaystyle
 \phantom{for~k} f_k = f_k - \sum_{i=1}^{k-1} (f_k.f_i)f_i\\
 
 Commentaires
 \left\{\!
	 \begin{array}{l}
	 $Si le mode $f_k$ orthogonalisé est très petit : $\\
	 ~~$C'est que le mode trouvé par le point fixe pouvait $\\
	 ~~~$être représenté dans la base des modes précédents. $\\
	 ~~$Alors on choisit de l'ignorer pour ne pas polluer la base.$\\
	 $Si le mode $f_k$ orthogonalisé satisfait un critère de norme : $\\
	 ~~$On l'ajoute à la base comme précédemment.$\\
	 $Donc :$\\
	 \end{array}
 \right.
 \\
 \phantom{for~k} $if $ \left( \left\Vert g_k  \right\Vert . 
 						\left\Vert f_k  \right\Vert \right)
 						< \epsilon\\
 \phantom{for~k end} k = k-1 \\
 \phantom{for~k} end\\
end
\end{array}  \\
\end{equation}

Sans les commentaires cela peut être plus lisible :
\begin{equation}{}
\begin{array}{l}
 for~k = 1$ à $ n \\
 \phantom{for~k} $initialiser $g_k \\
 \phantom{for~k} for~j=1$ à $j_{max} \\
 \phantom{for~k for~j } f_k = F (U_{k-1},g_k) \\
 \phantom{for~k for~j } g_k = G (U_{k-1},f_k) \\
 \phantom{for~k} end \\ 
 \phantom{for~k} for~i=1$ à $(k-1) \\
 \phantom{for~k for~i } g_i = g_i - (f_k.f_i)g_k\\
 \phantom{for~k} end\\
 \displaystyle
 \phantom{for~k} f_k = f_k - \sum_{i=1}^{k-1} (f_k.f_i)f_i\\
 \phantom{for~k} $if $ \left( \left\Vert g_k  \right\Vert . 
 						\left\Vert f_k  \right\Vert \right)
 						< \epsilon\\
 \phantom{for~k end} k = k-1 \\
 \phantom{for~k} end\\
end
\end{array}  \\
\end{equation}
