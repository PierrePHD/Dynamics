\documentclass[a4paper,12pt,titlepage]{article}

\usepackage[utf8]{inputenc}  % pour PC
\usepackage[francais]{babel}
\usepackage{fontenc}
%\usepackage[applemac]{inputenc}  % pour MAC 
\usepackage{amsfonts}
\usepackage{amsmath}
 \usepackage{amssymb}
\usepackage{graphicx}

\usepackage{layout}

\title{Livrable F1.1a MECASIF SP1, \\
Revue critique des méthodes applicables aux SP2 SP3 et SP4}

\author{David RYCKELYNCK (Armines), Pierre-Alain Boucard (ENS/LMT), \\
Lionel Fourment (Armines), Yvon Maday(UMPC/LJLL), \\
Fabrice Thouverez (ECL/LTDS)}

\oddsidemargin = 0pt
\textwidth = 450pt

%%%%%%%%%%%%%%%%%%%%%%%%%%%%%%%%%%%%%%%%%%%%%%%%%%%%%%%%%%%%%%%%%%%%%%%%%%%


\begin{document}

 %\layout

 \maketitle
 \sloppypar

%%%%%%%%%%%%%%%%%%%%%%%%%%%%%%%%%%%%%%%%%%%%%%%%%%%%%%%%%%%%%%%%%%%%%%%%%%%%%%%%%%%%%%%%%%%%%%%%%%%%%%%%%%%%%%%%%%%%%%%%%%%%%%%
%%%%%%%%%%%%%%%%%%%%%%%%%%%%%%%%%%%%%%%%%%%%%%%%%%%%%%%%%%%%%%%%%%%%%%%%%%%%%%%%%%%%%%%%%%%%%%%%%%%%%%%%%%%%%%%%%%%%%%%%%%%%%%%

\section{Mots clés à compléter}

Afin de répartir les contributions, je propose d'attribuer une liste de mots clés à des rédacteurs principaux de ce rapport. Ceux-ci pourront évidemment déléguer cette tâche. Je propose une première répartition de mots clés à compléter ou à modifier, ainsi qu'une liste de rédacteurs principaux.

\begin{itemize}
\item Pierre-Alain Boucard : méthode PGD, méthode LATIN, TVRC (éventuellement)
\item Yvon Maday : méthode des bases réduites, Algorithmes Glouton, méthodes EIM, GEIM, interpolation de bases réduites (éventuellement)
\item Fabrice Thouverez : réduction de modèle en mécanique des vibrations, linéaires et non linéaires.
\item Lionel Fourment : métamodèles et surfaces de réponse
\item David Ryckelynck : méthode POD, classification en méthodes a priori ou a posteriori, méthode APHR, gappy POD, Missing Point Estimation, Hyper-réduction
\end{itemize}

Je propose également de rédiger nos contribution en Latex avec des références insérées dans le fichier biblio.bib.

David Ryckelynck

\section{Exemple de texte}

La gappy POD a été proposée dans \cite{Everson-1995} puis utilisée dans \cite{{Willcox-2006},{Carlberg-2011a}}.

\section{Conclusion (brouillon)}

Les méthodes applicables au sous-programme SP2 sont :
\begin{itemize}
\item la PGD ...
\end{itemize}

Les méthodes applicables au sous-programme SP3 sont :
\begin{itemize}
\item l'EIM
\item la gappy POD
\item l'hyper-réduction
\item la POD hiérarchique...
\end{itemize}

Les méthodes applicables au sous-programme SP4 sont :
\begin{itemize}
\item l'EIM
\item la gappy POD
\item l'hyper-réduction
\item la POD hiérarchique ...
\end{itemize}

\bibliographystyle{unsrt}
\bibliography{biblio.bib}

\end{document}
