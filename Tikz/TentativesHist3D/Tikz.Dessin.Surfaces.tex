\documentclass{minimal}
\usepackage{pgfplots}
\usepackage{pgfplotstable}
\pgfplotsset{compat=newest}

\begin{document}

\begin{tikzpicture}[x={(0.866cm,-0.5cm)},y={(0.866cm,0.5cm)},z={(0cm,1cm)}]

\pgfplotstableread{DataTest.dat}{\firsttable}
  \pgfplotstablegetrowsof{DataTest.dat}
  \pgfmathtruncatemacro{\rows}{\pgfplotsretval-1}   
  

%    \addplot3[surf,red!50,mark=none,faceted color=black] coordinates {
%        (3,3,0) (3,3,0) (3,4,0) (3,4,0)
%        
%        (3,3,0) (3,3,1) (3,4,1) (3,4,0)
%
%        (4,3,0) (4,3,1) (4,4,1) (4,4,0)
%
%        (4,3,0) (4,3,0) (4,4,0) (4,4,0)
%    };


    \begin{axis}[
    view = {30}{35},% important to draw x,y in increasing order
    xmin = 0.5,
    ymin = 0.5,
    xmax = 4.5,
    ymax = 4.5,
    zmin = 0,
    zmax = 1,
    unbounded coords = jump,
    %colormap/jet
    %colormap={pos}{color(0cm)=(white); color(6cm)=(blue)}
    ]
    
\colorlet{redhsb}[hsb]{red}%
  \colorlet{bluehsb}[hsb]{blue}%
    \draw[->] (0,0,0) -- (1,0,0) node [black,left] {x};
        \draw[->] (0,0,0) -- (0,1,0) node [black,left] {y};
        \draw[->] (0,0,0) -- (0,0,1) node [black,left] {z};
\foreach \x in {1,...,3}{
	\foreach \y in {3,...,1}{
		\pgfmathparse{(\x-1)*3 + (2-(\y-1)) }
		\pgfplotstablegetelem{\pgfmathresult}{[index] 2}\of{\firsttable} 
	  	\let\z\pgfplotsretval 
%	  \pgfmathtruncatemacro{\teinte}{100-(\z*100)}
%	  	\colorlet{col}[rgb]{bluehsb!\teinte!redhsb}   
%	  	% inutile pour cette vue
%	  	   \fill[col] (\x-0.5,\y-0.5,\z) -- (\x-0.5,\y+0.5,\z) -- (\x-0.5,\y+0.5,0) -- (\x-0.5,\y-0.5,0) -- (\x-0.5,\y-0.5,\z);
%	  	   		\draw[black] (\x-0.5,\y-0.5,\z) -- (\x-0.5,\y+0.5,\z) -- (\x-0.5,\y+0.5,0) -- (\x-0.5,\y-0.5,0) -- (\x-0.5,\y-0.5,\z);
%	  	   \fill[col] (\x-0.5,\y+0.5,\z) -- (\x+0.5,\y+0.5,\z) -- (\x+0.5,\y+0.5,0) -- (\x-0.5,\y+0.5,0) -- (\x-0.5,\y+0.5,\z);
%	  	   		\draw[black](\x-0.5,\y+0.5,\z) -- (\x+0.5,\y+0.5,\z) -- (\x+0.5,\y+0.5,0) -- (\x-0.5,\y+0.5,0) -- (\x-0.5,\y+0.5,\z);
%	  	 % Visible pour cette vue  
%	  	    \fill[col] (\x+0.5,\y+0.5,\z) -- (\x+0.5,\y-0.5,\z) -- (\x+0.5,\y-0.5,0) -- (\x+0.5,\y+0.5,0) -- (\x+0.5,\y+0.5,\z);
%	  	    	\draw[black](\x+0.5,\y+0.5,\z) -- (\x+0.5,\y-0.5,\z) -- (\x+0.5,\y-0.5,0) -- (\x+0.5,\y+0.5,0) -- (\x+0.5,\y+0.5,\z);
%	  	    \fill[col] (\x+0.5,\y-0.5,\z) -- (\x-0.5,\y-0.5,\z) -- (\x-0.5,\y-0.5,0) -- (\x+0.5,\y-0.5,0) -- (\x+0.5,\y-0.5,\z);
%	  	     	\draw[black](\x+0.5,\y-0.5,\z) -- (\x-0.5,\y-0.5,\z) -- (\x-0.5,\y-0.5,0) -- (\x+0.5,\y-0.5,0) -- (\x+0.5,\y-0.5,\z);
%	  	  % Top face
%	  	   \fill[col] (\x-0.5,\y-0.5,\z) -- (\x-0.5,\y+0.5,\z) -- (\x+0.5,\y+0.5,\z) -- (\x+0.5,\y-0.5,\z) -- (\x-0.5,\y-0.5,\z);
%	  	    \draw[black] (\x-0.5,\y-0.5,\z) -- (\x-0.5,\y+0.5,\z) -- (\x+0.5,\y+0.5,\z) -- (\x+0.5,\y-0.5,\z) -- (\x-0.5,\y-0.5,\z);

%   		\edef\temp{%\colorlet{col}[rgb]{bluehsb!\z!redhsb}   
%   		\noexpand\fill[green] (\x-0.5,\y-0.5,\z) -- (\x-0.5,\y+0.5,\z) -- (\x+0.5,\y+0.5,\z) -- (\x+0.5,\y-0.5,\z) -- (\x-0.5,\y-0.5,\z);
	   \pgfmathtruncatemacro{\teinte}{100-(\z*100)}
	  	\colorlet{col}[rgb]{bluehsb!\teinte!redhsb} 
	  	\edef\temp{
      %\colorlet{col}[rgb]{bluehsb!\z!redhsb}    
   		\noexpand\addplot3 [surf,color=red,mark=none,faceted color=black]%
        coordinates{       		
        (\x-0.5,\y+0.5,0) (\x-0.5,\y+0.5,0) (\x-0.5,\y-0.5,0) (\x-0.5,\y-0.5,0)
        
        (\x-0.5,\y+0.5,0) (\x-0.5,\y+0.5,\z) (\x-0.5,\y-0.5,\z) (\x-0.5,\y-0.5,0)

        (\x+0.5,\y+0.5,0) (\x+0.5,\y+0.5,\z) (\x+0.5,\y-0.5,\z) (\x+0.5,\y-0.5,0)

        (\x+0.5,\y+0.5,0) (\x+0.5,\y+0.5,0) (\x+0.5,\y-0.5,0) (\x+0.5,\y-0.5,0)
        };
        }
        \temp
  		
%       		coordinates{% view = {120}{35}
%       		
%        (\x-0.5,\y-0.5,0) (\x-0.5,\y-0.5,0) (\x-0.5,\y+0.5,0) (\x-0.5,\y+0.5,0)
%        
%        (\x-0.5,\y-0.5,0) (\x-0.5,\y-0.5,\z) (\x-0.5,\y+0.5,\z) (\x-0.5,\y+0.5,0)
%
%        (\x+0.5,\y-0.5,0) (\x+0.5,\y-0.5,\z) (\x+0.5,\y+0.5,\z) (\x+0.5,\y+0.5,0)
%
%        (\x+0.5,\y-0.5,0) (\x+0.5,\y-0.5,0) (\x+0.5,\y+0.5,0) (\x+0.5,\y+0.5,0)
%        };

  		%\show\temp %-- uncomment this to see what the \temp macro does
   	}
}
%    \begin{axis}[
%    view = {30}{35},% important to draw x,y in increasing order
%    xmin = 0.5,
%    ymin = 0.5,
%    xmax = 4.5,
%    ymax = 4.5,
%    zmin = 0,
%    zmax = 1,
%    unbounded coords = jump,
%    %colormap/jet
%    %colormap={pos}{color(0cm)=(white); color(6cm)=(blue)}
%    ]
    \end{axis}
\end{tikzpicture}

\end{document}